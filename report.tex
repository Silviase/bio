%! Author = silviase
%! Date = 2020/07/19

% プリアンブル

%文字の大きさ、用紙の大きさ、段組
\documentclass[uplatex,11pt,a4j]{jarticle}
\bibliographystyle{jplain}

\usepackage{url}
\usepackage[dvipdfmx]{graphicx}
\usepackage{here}
\usepackage{listings, jlisting}

%表、図の番号形式
\def\thetable{\thesubsubsection.\arabic{table}}
\def\thefigure{\thesubsubsection.\arabic{figure}}

%間隔
\setlength\floatsep{0pt}
\setlength\textfloatsep{0pt}
\setlength\intextsep{0pt}
\setlength\abovecaptionskip{0pt}

\usepackage{geometry}
\usepackage{amsmath}
%余白
\geometry{left=10mm,right=10mm,top=10mm,bottom=15mm}

\usepackage{txfonts} %設定部分
\renewcommand{\lstlistingname}{結果}
\lstset{language=python,
    basicstyle={\ttfamily\small}, %書体の指定
    frame=tRBl, %フレームの指定
    framesep=10pt, %フレームと中身(コード)の間隔
    breaklines=true, %行が長くなった場合の改行
    lineskip=-0.5ex, %行間の調整
    tabsize=2 %Tabを何文字幅にするかの指定
}

\title{\vspace{-10mm}生命情報解析前半レポート}
\author {18B13863 前田航希}
\date {\today}

\begin{document}
%タイトルの出力
\maketitle

今回行ったのは塩基配列についての多重アラインメントの描画である.
最もnaiveなDynamic Programmingによるもので行ったが,
要素数が不定であっても,リソースさえ割くことが出来れば計算が可能なプログラムを作成した.

\section{アルゴリズムに関する説明}
要素数が固定であるものに関する多重アラインメントは,
Dynamic Programmingによく用いられる手法である多次元配列を要素数の分だけ用意したうえで,
それを埋めていくという手法を取る.しかしながら,今回は要素数や文字列を任意に設定できるという拘束条件をつけることにした.
そのため,要素数次元の多次元配列を作成しておくという手法は取ることが出来ない.

そこで今回はメモ化再帰を用いて高速化を図った.それぞれの文字列の何文字目までみたかをチェックするListを用意し,
これをDPの多次元配列に見立てる.一度最大値が確定した要素はListと最大値を対応させるDictionaryを用意し,二度以上計算することがないようにすることで,
計算量を$O(\max|L|^N)$まで削減している.


\section{ソースコード}
\lstinputlisting[caption=多重アラインメントをGUI表示させるプログラムのソースコード, label=プログラム]{alignment.py}

%参考文献の出力
\bibliography{main}
\end{document}